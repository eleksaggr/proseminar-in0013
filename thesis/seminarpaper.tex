% on Windows 10 bash(ubuntu 14.04), you need to:
% apt install texlive-latex-extra texlive-latex-recommended latex-xcolor latexmk make python-pygments
\documentclass[twocolumn]{article}

% T1 fonts not available on the default windows latex
%\usepackage[T1]{fontenc}
%\usepackage{lmodern}

\usepackage[pdftex]{graphicx}
\usepackage{xcolor}
\usepackage{url}
\setlength{\parindent}{0cm}
\setlength{\columnsep}{25pt}
\usepackage{minted}
\newminted{cpp}{mathescape,linenos,frame=single,numbersep=0.5mm}
\newminted{rust}{mathescape,linenos,frame=single,numbersep=0.5mm}
\definecolor{almostnogray}{rgb}{0.92,0.92,0.92}

\sloppy

% Your name
\author{Clemens Ruck \& Alex Egger\\ Technische Universit\"at M\"unchen}

\title{Proseminar ``The Rust Programming Language'' \\
       Summer Term 2017 \\
       {\bf Garbage Collection \& Reference Counting}
}

% Date of your talk
\date{\today}

\begin{document}

\maketitle

\begin{abstract}
% This paper aims to give a broad overview of existing memory management techniques
% and a quick introduction to the technique employed by the Rust programming language.
% Further it will demonstrate the different possibilities when working with memory in
% the Rust language.
\end{abstract}

% \section defines numbered parts of the paper with titles
% there also are \subsection and \subsubsection
\section{Introduction}

% Use labels to be able to refer to this position from somewhere else
\label{introduction}
\section{Garbage Collection}
\subsection{General principle}
\subsection{Reference Counting}
\textit{Reference counting} is a simple garbage collection algorithm, that uses reference counts to deallocate objects that are no longer referenced, and thus free the underlying memory.
Here reference count refers to an internal counter, that tracks the amount of active references to a specific object.
Everytime an active reference is destroyed the internal reference counter for that object is decremented.
Correspondingly when a new reference to an object is created, the reference counter is incremented, to reflect the number of active references.
When an objects reference count falls to zero, the object becomes inaccessible.
The memory used by such an inaccessible object can be then be freed safely.

Reference counting seems superior to any tracing algorithm, because of its simplicity, but there are a few caveats, that are not visibile at first glance.
Firstly when handling a large amount of objects a deletion may cause a large amount of objects to be freed in a chain reaction.
This chain reaction can then use up valuable processing time, resulting in a largely unresponsive application for a user.
This pitfall can be circumvented with the following approach: Whenever an object should be ordinarly deleted, it is instead added to a list of objects that are to be freed.
The list can then be processed at another point in time, effectively making the whole technique incremental.

A more severe problem is posed by \textit{reference cycles}, where multiple references reference each other, leading to a non-tree-like structure.
These cycles can not be reclaimed, since each object depdends on another being freed first.
These shortcomings render reference counting unsatisfactory in most contexts, but the following situations:
\begin{itemize}
        \item Reference loops are impossible.
        \item Modifications of references are infrequent.
        \item Memory constraints are very tight.
\end{itemize}
\subsection{Rust}
\section{Using memory in Rust}
\subsection{Allocating data on the heap}
\label{box}
It is common practice to store long-lived data on the heap section of the memory, as the lifetime of the data is then decoupled of the lifetime of its local function.
Rust allows allocating data on the heap by using the \mintinline{rust}{Box<T>} struct.
A \mintinline{rust}{Box} may be created by calling the \mintinline{rust}{Box::new()} method, as can be seen in figure \ref{box-example}.
It may seem like there is some kind of magic underneath the \mintinline{rust}{Box} type, but that would be contrary to Rusts goals.
In fact a \mintinline{rust}{Box} is only a wrapper around a raw pointer (see \ref{raw-pointer}), that points to the heap, and the size of the object that is stored.
A simple implementation of this \mintinline{rust}{Box} type is shown in figure \ref{box-definition}.
When not fully accustomed to the Rust type system one may be puzzled by the \mintinline{rust}{where}-clause in the struct definition.
This clause simply says the type that is stored in the \mintinline{rust}{Box} must have a fixed size and may not be a Dynamically Sized Type, like for example a Trait.

Now there is still the question of how the memory used by a \mintinline{rust}{Box} is freed.
The \mintinline{rust}{Box} struct implements the \mintinline{rust}{Drop} trait.
The documentation for this trait describes it as follows: ``The \mintinline{rust}{Drop} trait is used to run some code when a value goes out of scope. This is sometimes called a destructor''\cite{RustDoc-Drop}.
The \mintinline{rust}{Box} type uses this trait to free the memory reserved on the heap, when it goes out of scope.


\begin{figure}
\begin{rustcode}
fn main() {
    let i: u8 = 42;
    let b = Box::new(i);
}
\end{rustcode}
\vspace{-2em}
\caption{A program that stores an \mintinline{rust}{u8} on the heap.}
\label{box-example}
\end{figure}
\begin{figure}
\begin{rustcode}
pub struct Box<T> where T: ?Sized {
    pointer: *mut T,
    size: usize,
}
\end{rustcode}
\vspace{-2em}
\caption{A simplified example definition of the \mintinline{rust}{Box} type.}
\label{box-definition}
\end{figure}
\subsection{The \mintinline{rust}{Deref} trait}
\subsection{Reference Counting in Rust}
\subsection{Interior Mutability with Cell types}
``\textit{Interior Mutability} is a design pattern in Rust for allowing you to mutate data even though there are immutable references to that data, which would normally be disallowed by the borrowing rules. The interior mutability pattern involves using unsafe code inside a data structure to bend Rust's usual rules around mutation and borrowing.''\cite{RustBook-InteriorMutability}
\subsection{Raw pointers}
\label{raw-pointer}
Rust has two kinds of references that allow pointing to an arbitrary place in memory.
These references are called \textit{raw pointers}. 
The two types are \mintinline{rust}{*const T} and \mintinline{rust}{*mut T}.
How to create such raw pointers is shown in figure \ref{raw-pointer-new} and in figure \ref{raw-pointer-new2}.
The difference between the \mintinline{rust}{const} and the \mintinline{rust}{mut} variants is simply whether the value in the underlying memory address can be mutated through the pointer, or not.
In the example in figure \ref{raw-pointer-new} we created two pointers to the same address.
One begin a \mintinline{rust}{const *u32} and the other a \mintinline{rust}{mut *u32}.
If we were to create a \mintinline{rust}{&u32} and a \mintinline{rust}{&mut u32} we would get a compiler error, but when working with raw pointers the usual borrowing rules don't apply.
\begin{figure}
\begin{rustcode}
let count: u32 = 65;
let ptr = &count as *const u32;
let mut_ptr = &mut count as *mut u32;
\end{rustcode}
\vspace{-2em}
\caption{Creating two raw pointers from a reference to an \mintinline{rust}{u32}.}
\label{raw-pointer-new}
\end{figure}
\begin{figure}
\begin{rustcode}
let address = 0x123456;
let ptr = address as *mut u32;
\end{rustcode}
\vspace{-2em}
\caption{Creating a raw pointer from an address.}
\label{raw-pointer-new2}
\end{figure}
\subsection{Writing unsafe Code}
\label{unsafe}
In some situations a programmer may need to write code, that the static guarantees of the compiler would reject.
Such situations may for example occur whenever raw pointers (see \ref{raw-pointer}) are used to work around some restriction of Rusts borrowing mechanism.
Another possibility is whenever Rust tries to interact with hardware directly, since hardware does not adhere to Rusts rules.
The \mintinline{rust}{unsafe} keyword can be used to communicate to the compiler, that the designated block contains 'unsafe' code.
'Unsafe' here means exactly the following behaviours:
\begin{enumerate}
        \item Dereferencing a raw pointer.
        \item Calling an unsafe function or method.
        \item Accessing or modifying a mutable static variable.
        \item Implementing an unsafe trait.
\end{enumerate}
\cite{RustBook-Unsafe}.
\paragraph{Calling an unsafe function or method} is unsafe behaviour, because these unsafe functions generally assume that some invariant is true before they are called.
A programmer may have unknowingly broken the invariant before calling the function, resulting in a memory violation or other unwanted behaviour.
An example of this can be seen in figure \ref{unsafe-function}, in which a function called \mintinline{text}{add_three} takes a mutable raw pointer (see \ref{raw-pointer}), offsets it by one, derefences it, and adds three to it.
Obviously this is not memory-safe when the pointer does not own the element after it in memory, and as such the function must be marked \mintinline{rust}{unsafe}.
The \mintinline{rust}{unsafe} keyword in the function definition additonally acts like an \mintinline{rust}{unsafe} block around the whole function body, which is the reason the dereferencing of the pointer in figure \ref{unsafe-function} is not wrapped in an \mintinline{rust}{unsafe} block.
\begin{figure}
\begin{rustcode}
unsafe fn add_three(ptr: *mut u8) {
        *(ptr.offset(1)) += 3;
}
\end{rustcode}
\vspace{-2em}
\caption{An unsafe function, that adds three to a pointer offset by one.}
\label{unsafe-function}
\end{figure}
\paragraph{Accessing or updating a static mutable variable} is defined to be unsafe behaviour.
Static mutable variables are comparable to global state in other programming languages.
Hoare describes Rust as having a ``safe concurrency model''\cite{HoareInterview}.
This is especially visibile when looking at the concept of borrowing, where Rust enforces safe concurrency through the restrictions on \mintinline{rust}{&T} and \mintinline{rust}{&mut T} references.
Additionally it is Rusts responsibility to ensure data races on global state can not happen.
Infact the Rust compiler will not allow accessing or updating such global state, without declaring the operation
as \mintinline{rust}{unsafe}, after which it is the programmers responsibility to ensure correct handling of data races.
An example of updating a static mutable variable can be seen in figure \ref{unsafe-staticmut}.
\begin{figure}
\begin{rustcode}
static mut COUNT: u8 = 0;

fn main() {
    unsafe {
        COUNT = 1;
    }
}
\end{rustcode}
\vspace{-2em}
\caption{Updating a static mutable variable in an \mintinline{rust}{unsafe} block.}
\label{unsafe-staticmut}
\end{figure}
\paragraph{Implementing an unsafe trait}
\section{Conclusion}
\section{Future work}

% \emph{Latex} is the system of choice for scientific publications. Papers
% are expected to be delivered in Latex form, scientific publishers
% even require you to make use of their templates \cite{springer,acm}.
% We thus strongly encourage students to get to grips with Latex. As
% programmers, You will most likely enjoy the workflow of a text
% processor like Latex anyway.

% The following is meant as a demonstration of the capabilities and
% the source code, that is necessary to produce this output. You may
% use the source code of this seminar paper as a template, if you wish.

% %start demonstration

% First, we want to refer to the figures and the introduction.
% See Fig.\ref{TUM} for the first floating figure with column width,
% and Fig.\ref{Fig2} for the one using the full page width.
% And here, we want to put a reference to the introduction which is
% Section \ref{introduction}.

% In translating this template from German to English, I decided to
% stop here. There is not really much to get from the German text
% following. Anything Latex-related can also be looked up on the
% net. There is a {\it huge} number of tutorials, and so on.

% Please do not use to much different font sizes and styles. It should
% be completely enough to go to {\em italic mode} for emphasizing something,
% such as newly introduced terms.
% You can refer to other parts of your paper (e.g. see Sec.~\ref{introduction}).
% Quoting in Latex is done ``this way''.
% Further, you may have problems with punctation characters.
% Most of them just need to be prefixed by a backslash, for others you may
% temporarily switch to math mode:
% \$ \& \% \# \{ \} [ ] \_ @ \S $<$ $>$ $\backslash$ @ \textasciitilde /

% Talking about math mode: you can do some very nice things this way:
% \begin{equation}
% a^2 + b^2 = c^2
% \label{Pythagoras}
% \end{equation}
% Again, refering to this equation is easy (see Eq.~\ref{Pythagoras}).
% If you do not need numbering for equations, use the {\em displaymath}
% environment:
% \begin{displaymath}
% x_{1,2} = \frac{-b \pm \sqrt{b^2-4ac}}{2a}\\
% \end{displaymath}
% Short equations simply can be used within the regular text flow, such
% as with $x \to \infty$. Obviously, math is fun with Latex.


% \subsection*{Enumerations}

% Enumerations using bullet points:

% \begin{itemize}
% 	\item this is the first item of this list of interesting facts,
% 	\item second item,
% 	\item and the last one.
% \end{itemize}

% They also can be numbered:

% \begin{enumerate}
% 	\item item one,
% 	\item item two,
% 	\item item three.
% \end{enumerate}

% As shown, numbers always should be written out in the text, unless the
% belong to a title or a formula.

% \section{Literature}

% At the end of your paper, you should have a nice list of used
% literature. For scientific papers, this actually is needed. You always
% use other works as base for your own. Usually, you are not the only
% one thinking about a given difficult problem, so there is always
% related work which {\em must} be cited if known to the author.

% Further, if you want to copy relevant sentences from an
% original paper, you {\em have} to cite them correctly, for example
% in this way:
% \begin{quote}
% 	``I think there is a world market for maybe five computers.''
% 	(T.J. Watson, IBM, 1943)
% \end{quote}
% However, in computer science, direct citations are uncommon if not
% even considered bad style -- with the notable exception of lemmas
% and theorems.

% Much more common is the indirect citation. Here, the cited work
% (especially all the regular text) must be
% written/phrased by you. If you write about some results or fact
% stated in another paper, you should refer to it.
% The `Analytical Engine'' --- a mechanical calculation machine ---
% created by Charles Babbage in the year 1838 was based on the decimal
% system
% % use \cite to refer to papers from seminarpaper.bib
% % this file is processed by bibtex, and it automatically adds numbering
% \cite{Brom98}.

% You can find new sources for scientific topics quite comforatbly via
% DBLP Computer Science Bibliography \cite{DBLP}. This site not only
% acts as a registry for almost all publications in computer scientific
% journals and conferences, but also provides You with correctly
% formated \emph{bibtex}-entries, ready to be integrated into your
% seminar papers \emph{bib}-file.

% \section{Figures and Tables}

% No need to understand the following text.

% Figures can span either one column (see Fig.~\ref{TUM}) or the full
% page width (see Fig.~\ref{Fig2}).
% Latex automatically tries to find the best place for these floating
% figures. To influence that, you may move the figure a bit to the front
% of your text.
% As can be seen in Fig.~\ref{TUM}, using raster images usually results in
% quite bad quality. Better use vector formats: draw the figures with
% {\em inkscape} \cite{inkscape}, and save them as PDF or SVG. As example of
% this procedure, see Fig.~\ref{Fig2}).

% % Narrow tables (just one column) have no * at the end
% \begin{table*}
% % label is for this table
% \label{Tab1}
% \begin{center}
% % arguments:
% % c = center
% % l = left
% % r = right (z. B. for cash)
% % p = columns with fixed width, using block layout
% % | = vertical line
% \begin{tabular}{|l|p{2cm}|c|c|c|c|c|r|}
% % horizontal line
% \hline
% % & means: next column
% % \\ means: next row
% 	& Column 1 & Column 2 & Column 3 & Column 4& Column 5& Column 6& Amount\\
% \hline
% Row 1 & This column has a maximal width of 2 cm.& X & X& X& X& X& 126,00\\
% \hline
% Row 2 & & \multicolumn{3}{p{5cm}|}{This entry occupies three columns.}& X &X & 8,00\\
% \hline
% \multicolumn{7}{|l}{Sum} &134,00\\
% \hline
% \end{tabular}
% \caption{For this layout, we want table captions to be {\em below} the actual table.}
% \end{center}
% \end{table*}

% Similar to figures, tables can be referred to in the text (see Tab.~\ref{Tab1}).
% However, sometimes it is useful to embed tables directly in the regular
% text flow:

% \begin{center}
% \begin{tabular}{|c|c|c|}
% \hline
% 	& Column 1 & Column 2 \\
% \hline
% Row 1 & & \\
% Row 1 & & \\
% \hline
% \end{tabular}
% \end{center}


% \section{Summary}
% \label{summary}

% The summary shortly repeats the core ideas and results from the
% previous text. If the reader has problems understanding the summary
% he knows that he should go back to the relevant sections.
% Thus, the last section should consist of:

% \begin{itemize}
% 	\item a summary,
% 	\item an evaluation of what was done, importance of this work,
% 	\item what is left, what still needs to be done,
%         \item short outlook into the future.
% \end{itemize}

% Last but not least, we can explain anything missing yet in the evaluation
% done in this paper. This allows to refer to what readers can expect from
% authors in the future.

% % Put citations from bibtex into References section which were not
% % explicity cited.
% % \nocite{robotron,
% % stonx,vice,650sim,herculessim,zib,4004,thermal1,thermal2,rojas,neumann,
% % neumann1void,neumann2void,neumann3,neumann4,Siewiorek,AmBlBr,Blaauw:1997:CAC,ChPB97,Brom98,Clym93,Buhu99,
% % Edwa01,Nill99,ABC+90,Rama91,Heid97,Kist95,Klot03,LMcCL92,LFS+92,LMDD92,Vlec03,Cray77,top500,650,Amda67,
% % Arla88,ODLD86,Hick04,Walk04}
% % 

\bibliographystyle{plain}
% Literature sources are to be found in seminarpaper.bib
\bibliography{seminarpaper} 
\end{document}
