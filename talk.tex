\documentclass{beamer}
\usepackage[utf8]{inputenc}

\graphicspath{{svg/}}

\usefonttheme[onlymath]{serif}
\usefonttheme{structurebold}
\usetheme{Boadilla}
\usecolortheme{rose}
\beamertemplatebookbibitems
\setbeamertemplate{navigation symbols}{}

\setbeamertemplate{footline}
{
	\leavevmode%
	\hbox{%
		\begin{beamercolorbox}[wd=.333333\paperwidth,ht=2.25ex,dp=1ex,center]{author in head/foot}%
			\usebeamerfont{author in head/foot}\insertshortauthor
		\end{beamercolorbox}%
		\begin{beamercolorbox}[wd=.333333\paperwidth,ht=2.25ex,dp=1ex,center]{title in head/foot}%
			\usebeamerfont{title in head/foot}\insertsection
		\end{beamercolorbox}%
		\begin{beamercolorbox}[wd=.333333\paperwidth,ht=2.25ex,dp=1ex,right]{date in head/foot}%
			\usebeamerfont{date in head/foot}\insertsubsection\hspace*{2em}
			\insertframenumber{} / \inserttotalframenumber\hspace*{2ex} 
		\end{beamercolorbox}}%
	\vskip0pt%
}

\usepackage[absolute,overlay]{textpos}
\textblockorigin{1em}{2em} % coordinates of zero/zero
\setlength{\TPHorizModule}{.10\textwidth}   % units are 1/10th of a page width
\setlength{\TPVertModule}{.10\textheight}   % units are 1/10th of a page height
\newcommand{\absolute}[4]{
	\begin{textblock}{0.1}(#1,#2) % {horizontale Ausdehnung}(linksoben_x, linkoben_y) ?
		\pgfimage[height=#3]{#4}
	\end{textblock}
}
\usepackage{graphicx,svg,colortbl,etex,helvet,tikz,textcomp}

%Source Code highlighting
\usepackage{minted}
\usemintedstyle{colorful}
%Background for minted source code
\definecolor{almostnogray}{rgb}{0.92,0.92,0.92}
\definecolor{dgreen}{rgb}{0.1,0.5,0.1}
\definecolor{dgrey}{rgb}{0.5,0.5,0.5}
\definecolor{dred}{rgb}{0.5,0.1,0.1}

\newcommand{\haekchen}{{\Large\color{dgreen}\checkmark}}
\newcommand{\attention}{{\scalebox{.75}{\bf\input{svg/attention.pdf_tex}}}\ }
\newcommand{\conclude}{{\color{dred}$\leadsto\;$}}

% Math abbrevs etc...
\renewcommand{\O}[1]{{{\cal O}({#1})}}

%%%%%%%%%%%%%%%%%%%%%%%%%%%%%%%%%%%%%%%%%%%%%%%%%%%%%%%%%%%%%%%%%%%%%%%%%%%%%
%%%
%%% ABAKUS, use with \ABAKUS
%%%
%%%%%%%%%%%%%%%%%%%%%%%%%%%%%%%%%%%%%%%%%%%%%%%%%%%%%%%%%%%%%%%%%%%%%%%%%%%%%

\def\ABAKUS{\begingroup\offinterlineskip

% Parameter (\"anderbar)
\dimen0=24pt            % Gesamtgr\"o{\ss}e
\dimen1=0.01\dimen0 % Strichdicke
%\dimen1=.8pt
\font\kugelfont=cmbsy5 scaled \magstephalf   % 1,2 facher cmbsy font
\def\kugel{{\kugelfont\char'16}}%

% Hilfsgr\"o{\ss}en
\dimen2=\dimen0\advance\dimen2by-6\dimen1\divide\dimen2by5%Spaltenbreite
\def\h{\hskip\dimen2}%
\dimen3=\dimen0\advance\dimen3by-3\dimen1\divide\dimen3by7%Zeilenh\"ohe
\setbox0\hbox to\dimen1{\hss\kugel\hss}%
\dimen4=\dimen3\advance\dimen4by-\ht0\advance\dimen4by-\dp0
  \divide\dimen4by2%Reststrichl\"ange \"uber und unter der Kugel
\def\restline{\hrule height\dimen4 depth0pt width\dimen1}%
\def\O{\vbox{\restline\copy0\restline}}%
\def\N{\vrule height\dimen3 depth0pt width\dimen1}%
% Abakus Tabelle
\vtop{\hrule height0pt depth\dimen1
      \hbox{\N\h\O\h\N\h\N\h\N\h\N}
      \hbox{\N\h\N\h\O\h\O\h\O\h\N}
      \hrule height\dimen1
      \hbox{\N\h\O\h\O\h\O\h\O\h\N}
      \hbox{\N\h\N\h\O\h\N\h\O\h\N}
      \hbox{\N\h\O\h\O\h\O\h\N\h\N}
      \hbox{\N\h\O\h\O\h\O\h\O\h\N}
      \hbox{\N\h\O\h\N\h\O\h\O\h\N}
      \hrule height\dimen1
     }
\endgroup}

%%%%%%%%%%%%%%%%%%%%%%%%%%%%%%%%%%%%%%%%%%%%%%%%%%%%%%%%%%%%%%%%%%%%%%%%%%%%%
%%%
%%% TUM Logo, use with \TUM{1cm}
%%%
%%%%%%%%%%%%%%%%%%%%%%%%%%%%%%%%%%%%%%%%%%%%%%%%%%%%%%%%%%%%%%%%%%%%%%%%%%%%%

\def\TUM#1{% von Gunnar Teege (leicht ge\"andert (siehe Kommentar))
%Argument gibt die Breite an
\dimen0=#1\dimen0=.018\dimen0%(war.012)% (war .018)
\dimen1=#1\dimen1=.1143\dimen1%
\dimen2=#1\dimen2=.419\dimen2%
\dimen3=#1\dimen3=.0857\dimen3%
%
\dimen4=\dimen1\advance\dimen4 by-\dimen0%
\setbox1=\vbox{\hrule width\dimen0 height\dimen4 depth0pt}%
\advance\dimen4 by\dimen2%
\setbox8=\vbox{\hrule width\dimen0 height\dimen4 depth0pt}%
\advance\dimen4 by-\dimen2\advance\dimen4 by-\dimen0%
\setbox4=\vbox{\hrule width\dimen4 height\dimen0 depth0pt}%
\advance\dimen4 by\dimen1\advance\dimen4 by\dimen3%
\setbox6=\vbox{\hrule width\dimen4 height\dimen0 depth0pt}%
\advance\dimen4 by\dimen3\advance\dimen4 by\dimen0%
\setbox9=\vbox{\hrule width\dimen4 height\dimen0 depth0pt}%
\advance\dimen4 by\dimen1%
\setbox7=\vbox{\hrule width\dimen4 height\dimen0 depth0pt}%
\dimen4=\dimen3%
\setbox5=\vbox{\hrule width\dimen4 height\dimen0 depth0pt}%
\advance\dimen4 by-\dimen0%
\setbox2=\vbox{\hrule width\dimen4 height\dimen0 depth0pt}%
\dimen4=\dimen2\advance\dimen4 by\dimen0%
\setbox3=\vbox{\hrule width\dimen0 height\dimen4 depth0pt}%
%
\setbox0=\vtop{\vskip0pt% (war \vbox und ohne \vskip0pt)
\hbox{\box9\lower\dimen2\copy3\lower\dimen2\copy5\lower\dimen2\copy3\box7}%
\kern-\dimen2\nointerlineskip%
\hbox{\raise\dimen2\box1\raise\dimen2\box2\copy3\copy4\copy3%
\raise\dimen2\copy5\copy3\box6\copy3\raise\dimen2\copy5\copy3\copy4\copy3%
\raise\dimen2\box5\box3\box4\box8}}%
\leavevmode\box0}


%%%%%%%%%%%%%%%%%%%%%%%%%%%%%%%%%%%%%%%%%%%%%%%%%%%%%%%%%%%%%%%%%%%%%%%%%%%%%
%%%
%%% TUM Header, use with \tumheader
%%%
%%%%%%%%%%%%%%%%%%%%%%%%%%%%%%%%%%%%%%%%%%%%%%%%%%%%%%%%%%%%%%%%%%%%%%%%%%%%%

\def\tumheader{
%\dimen5=9.1cm% \ / Diese drei Gr\"o{\ss}en
\dimen5=0.45\textwidth

\hbox to\textwidth{\ABAKUS\hfil\hbox to\dimen5{\vtop{\vskip0pt
    \hbox to\dimen5
        {\scriptsize TECHNISCHE\ UNIVERSIT{\kern-.1em}\"A{\kern-.1em}T\ M\"UNCHEN}
    \vskip2mm
    \hbox to\dimen5
        {\small FAKULT{\kern-.1em}\"A{\kern-.1em}T\ F\"UR\ INFORMATIK}
}}\hfil\TUM{1.5cm}}}


% title
\title{Garbage Collection \& Reference Counting}
\author{Memory management in Rust}
\institute[TUM]{
	\vspace{-15em}\tumheader\vspace{18em}}
\date{Clemens Ruck \& Alex Egger \\ Summer Term 2017}

\begin{document}
\setbeamertemplate{footline}{}
\maketitle
\begin{frame}
	\frametitle{Overview}
	\begin{textblock}{6}(0,0)
		\begin{alertblock}{Methods of Memory Management}
			\begin{enumerate}
				\item Shortcomings of Manual Memory Management
				\item Garbage Collection
				\item Rust's approach
			\end{enumerate}
		\end{alertblock}
		\begin{alertblock}{Memory Management in Rust}
			\begin{enumerate}
				\item Stack Allocation
				\item Heap Allocation
				\item Reference Counting in Rust
				\item The 'unsafe' keyword
			\end{enumerate}
		\end{alertblock}
	\end{textblock}
\end{frame}
\begin{frame}[fragile]
	\frametitle{Problems - Memory Leaks}
	\begin{minted}[tabsize=4]{c}
int main(void) {
	while(1) {

		// Allocate some amount of memory on the heap.
		char *c;
		if(!(c = malloc(20 * sizeof(char)))) {
			perror("Could not allocate memory on heap.");
			return 1;
		}

		// Do something with allocated memory.

		// Memory is never freed, and can never be reclaimed
		// by the system.
	}
	return 0;
}
	\end{minted}
\end{frame}
\begin{frame}[fragile]
	\frametitle{Problems - Double Free}
	\begin{minted}[tabsize=4]{c}
int main(void) {
	// Allocate some memory just like before.
	char *c = malloc(10);

	do_something(c);

	// Do something again, but the memory was already freed!
	do_something(c);
}

void do_something(char *c) {
	// Do something with the memory here.

	// Then free the memory.
	free(c);
}
	\end{minted}
\end{frame}
\begin{frame}[fragile]
	\frametitle{Problems - Use after Free}
	\begin{minted}[tabsize=4]{c}
int main(void) {
	// Allocate memory, what a surprise.
	char *c = malloc(10);

	// Do something and then free the memory.
	free(c);

	// The memory now doesn't belong to us anymore, so this 
	// will result in a segmentation fault.
	*c++;
}
	\end{minted}
\end{frame}
\begin{frame}		
	\frametitle{Garbage Collection}
	\textbf{Garbage Collection} is a form of automatic memory management. It attempts to reclaim memory used by objects that are no longer in use.
\end{frame}
\begin{frame}
	\frametitle{Example - Mark \& Sweep}
	\begin{figure}
		\centering
		\def\svgwidth{230pt}
		\input{svg/mark_sweep_setup.pdf_tex}
		\caption{A graph-represenation of alive objects.}
	\end{figure}
\end{frame}
\begin{frame}
	\frametitle{Example - Mark \& Sweep}
	\begin{figure}
		\centering
		\def\svgwidth{230pt}
		\input{svg/mark_sweep_mark.pdf_tex}
		\caption{The 'Mark' stage of the algorithm.}
	\end{figure}
\end{frame}
\begin{frame}
	\frametitle{Example - Mark \& Sweep}
	\begin{figure}
		\centering
		\def\svgwidth{230pt}
		\input{svg/mark_sweep_sweep.pdf_tex}
		\caption{The 'Sweep' stage of the algorithm.}
	\end{figure}
\end{frame}
\begin{frame}
	\frametitle{Memory management in Rust}
	Rust employs \textbf{ownership rules} to ensure memory safety.
	Values are stack-allocated per default. To allocate memory on the heap one can use \mintinline{rust}{Box<T>}.
\end{frame}
\begin{frame}[fragile]
	\frametitle{Stack Allocation}
	\begin{columns}[T, c]
		\begin{column}{0.48\textwidth}
			After line 2: \\~\\
			\begin{tabular}{| l | l | l |}
				\hline
				Address & Name & Value \\ \hline
				0 & x & 42 \\ \hline
			\end{tabular}
		\end{column}
		\begin{column}{0.48\textwidth}
			\begin{minted}[linenos, tabsize=4, frame=single]{rust}
fn main() {
    let x = 42;
    other();
}

fn other() {
	let y = 27;
	let z = 99;
}
			\end{minted}
		\end{column}
	\end{columns}
\end{frame}
\begin{frame}[fragile]
	\frametitle{Stack Allocation}
	\begin{columns}[T, c]
		\begin{column}{0.48\textwidth}
			After line 8: \\~\\
			\begin{tabular}{| l | l | l |}
				\hline
				Address & Name & Value \\ \hline
				2 & z & 99 \\ \hline
				1 & y & 27 \\ \hline
				0 & x & 42 \\ \hline
			\end{tabular}
		\end{column}
		\begin{column}{0.48\textwidth}
			\begin{minted}[linenos,tabsize=4,frame=single]{rust}
fn main() {
	let x = 42;
	other();
}

fn other() {
	let y = 27;
	let z = 99;
}
			\end{minted}
		\end{column}
	\end{columns}
\end{frame}
\begin{frame}[fragile]
	\frametitle{Stack Allocation}
	\begin{columns}[T, c]
		\begin{column}{0.48\textwidth}
			After line 3: \\~\\
			\begin{tabular}{| l | l | l |}
				\hline
				Address & Name & Value \\ \hline
				0 & x & 42 \\ \hline
			\end{tabular}
		\end{column}
		\begin{column}{0.48\textwidth}
			\begin{minted}[linenos,tabsize=4,frame=single]{rust}
fn main() {
	let x = 42;
	other();
}

fn other() {
	let y = 27;
	let z = 99;
}
			\end{minted}
		\end{column}
	\end{columns}
\end{frame}
\begin{frame}[fragile]
	\frametitle{Heap Allocation}
	\begin{columns}[T, c]
		\begin{column}{0.48\textwidth}
			\begin{tabular}{| l | l | l |}
				\hline
				Adress & Name & Value  \\ \hline
				1 & y & ??? \\ \hline
				0 & x & 42 \\ \hline
			\end{tabular}
		\end{column}
		\begin{column}{0.48\textwidth}
			\begin{minted}[linenos, tabsize=4, frame=single]{rust}
fn main() {
	let x = 42;
	let y = Box::new(39);
}
			\end{minted}
		\end{column}
	\end{columns}
\end{frame}
\begin{frame}[fragile]
	\frametitle{Heap Allocation}
	\begin{columns}[T, c]
		\begin{column}{0.48\textwidth}
			\begin{tabular}{| l | l | l |}
				\hline
				Adress & Name & Value  \\ \hline
				ffff & & 39 \\ \hline
				&   & \\ \hline
				1 & y & \\ \hline
				0 & x & $->$ ffff \\ \hline
			\end{tabular}
		\end{column}
		\begin{column}{0.48\textwidth}
			\begin{minted}[linenos, tabsize=4, frame=single]{rust}
fn main() {
	let x = 42;
	let y = Box::new(39);
}
			\end{minted}
		\end{column}
	\end{columns}
\end{frame}
\begin{frame}
	\frametitle{Comparison: Heap vs. Stack}
	\begin{enumerate}
		\item Managing the Stack is trivial
		\item Managing the Heap is non-trivial
		\item Having stack-allocation as the default allows easier reasoning about the lifetimes of objects
	\end{enumerate}
\end{frame}
\begin{frame}
	\frametitle{Reference Counting}
	References can be represented as a directed graph, where the vertices are objects and there is an edge between the nodes if one holds a reference to the other.
	\begin{figure}
		\def\svgwidth{120pt}
		\LARGE
		\input{svg/rc_setup.pdf_tex}
	\end{figure}
\end{frame}
\begin{frame}
	\frametitle{Reference Counting}
	Everytime a new reference to an object is created, the \textbf{reference counter} is incremented.
	\begin{figure}
		\def\svgwidth{120pt}
		\LARGE
		\input{svg/rc_new.pdf_tex}
	\end{figure}
\end{frame}
\begin{frame}
	\frametitle{Reference Counting}
	When an object is freed all references it holds are freed too, and the respective reference counters are decremented.
	\begin{figure}
		\def\svgwidth{120pt}
		\LARGE
		\input{svg/rc_delete.pdf_tex}
	\end{figure}
\end{frame}
\begin{frame}
	\frametitle{Reference Counting}
	Objects can only be freed, when their \textbf{reference count is 0!} Otherwise we'll end up with dangling references.
	\begin{figure}
		\def\svgwidth{120pt}
		\LARGE
		\input{svg/rc_delete_bad.pdf_tex}
	\end{figure}
\end{frame}
\begin{frame}
	\frametitle{Limitations}
	\attention Reference cycles can never be reclaimed!
	\begin{figure}
		\def\svgwidth{100pt}
		\input{svg/rc_cycle.pdf_tex}
	\end{figure}
	This can be solved by the use of a dedicated incremental garbage collector, that specifically targets reference cycles.
	An other approach is to simply disallow cycles in your data structure.
\end{frame}
\begin{frame}[fragile]
	\frametitle{Reference Counting - Example}
	\begin{minted}[tabsize=4, frame=single]{rust}
struct Owner {
	name: String,
	// Fields...
}	

struct Car {
	owner: Rc<Owner>,
	// Fields...
}
	\end{minted}
\end{frame}
\begin{frame}[fragile]
	\frametitle{Reference Counting - Example}
	\begin{minted}[tabsize=4, frame=single]{rust}
fn main() {
	let owner = Rc::new(Owner{
		name: "Lars",
		// Fields...
	});

	let car = Car {
		owner: owner.clone(),
		// Fields...
	}

	let car2 = Car {
		owner: owner.clone(),
		// Fields...
	}
	// ...
}
	\end{minted}
\end{frame}
\begin{frame}[fragile]
	\frametitle{Reference Counting - Example}
	\begin{minted}[tabsize=4, frame=single]{rust}
fn main() {
	// ...

	// Drop the local variable 'owner'
	drop(owner);

	// This will still work, 
	// since the owner binding survives using Rc!
	println!("{}", car.owner.name);
}
	\end{minted}
\end{frame}

\begin{frame}
	\frametitle{The 'unsafe' keyword}
	The \mintinline{rust}{unsafe} keyword allows:
	\begin{enumerate}
		\item Accessing or updating of a static mutable variable
		\item Dereferencing of a raw pointer
		\item Calling of other unsafe functions
	\end{enumerate}
\end{frame}
\begin{frame}[fragile]
	\frametitle{Raw Pointers}
	Two types of raw pointers:
	\begin{enumerate}
		\item \mintinline{rust}{*const T}
		\item \mintinline{rust}{*mut T}
	\end{enumerate}
	\vspace{1cm}
	Raw pointers have no lifetime or ownership. The only guarantee provided is they cannot be dereferenced except in code marked as \mintinline{rust}{unsafe}.
\end{frame}
\begin{frame}[fragile]
	\frametitle{Example}
	\begin{minted}[linenos, tabsize=4, frame=single]{rust}
fn main() {
	let i: u32 = 77;

	// Creating a raw pointer in safe code is 
	// perfectly acceptable.
	let x = &i as *const u32; 

	// Dereferencing a raw pointer in 
	// safe code is not allowed!
	let y = *x;

	// Once we marked the code as unsafe, 
	// the compiler let's us do it.
	unsafe {
		let z = *x;
	}
}
	\end{minted}
\end{frame}
\end{document}
