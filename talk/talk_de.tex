\documentclass{beamer}
\usepackage[utf8]{inputenc}

\graphicspath{{svg/}}

\usefonttheme[onlymath]{serif}
\usefonttheme{structurebold}
\usetheme{Boadilla}
\usecolortheme{rose}
\beamertemplatebookbibitems
\setbeamertemplate{navigation symbols}{}

\setbeamertemplate{footline}
{
	\leavevmode%
	\hbox{%
		\begin{beamercolorbox}[wd=.333333\paperwidth,ht=2.25ex,dp=1ex,center]{author in head/foot}%
			\usebeamerfont{author in head/foot}\insertshortauthor
		\end{beamercolorbox}%
		\begin{beamercolorbox}[wd=.333333\paperwidth,ht=2.25ex,dp=1ex,center]{title in head/foot}%
			\usebeamerfont{title in head/foot}\insertsection
		\end{beamercolorbox}%
		\begin{beamercolorbox}[wd=.333333\paperwidth,ht=2.25ex,dp=1ex,right]{date in head/foot}%
			\usebeamerfont{date in head/foot}\insertsubsection\hspace*{2em}
			\insertframenumber{} / \inserttotalframenumber\hspace*{2ex}
		\end{beamercolorbox}}%
	\vskip0pt%
}

% \makeatletter
% \renewcommand{\fnum@figure}{Grafik \thefigure}
% \makeatother

\usepackage[absolute,overlay]{textpos}
\textblockorigin{1em}{2em} % coordinates of zero/zero
\setlength{\TPHorizModule}{.10\textwidth}   % units are 1/10th of a page width
\setlength{\TPVertModule}{.10\textheight}   % units are 1/10th of a page height
\newcommand{\absolute}[4]{
	\begin{textblock}{0.1}(#1,#2) % {horizontale Ausdehnung}(linksoben_x, linkoben_y) ?
		\pgfimage[height=#3]{#4}
	\end{textblock}
}
\usepackage{graphicx,svg,colortbl,etex,helvet,tikz,textcomp}

%Source Code highlighting
\usepackage{minted}
\usemintedstyle{colorful}
%Background for minted source code
\definecolor{almostnogray}{rgb}{0.92,0.92,0.92}
\definecolor{dgreen}{rgb}{0.1,0.5,0.1}
\definecolor{dgrey}{rgb}{0.5,0.5,0.5}
\definecolor{dred}{rgb}{0.5,0.1,0.1}

\newcommand{\haekchen}{{\Large\color{dgreen}\checkmark}}
\newcommand{\attention}{{\scalebox{.75}{\bf\input{svg/attention.pdf_tex}}}\ }
\newcommand{\conclude}{{\color{dred}$\leadsto\;$}}

% Math abbrevs etc...
\renewcommand{\O}[1]{{{\cal O}({#1})}}

%%%%%%%%%%%%%%%%%%%%%%%%%%%%%%%%%%%%%%%%%%%%%%%%%%%%%%%%%%%%%%%%%%%%%%%%%%%%%
%%%
%%% ABAKUS, use with \ABAKUS
%%%
%%%%%%%%%%%%%%%%%%%%%%%%%%%%%%%%%%%%%%%%%%%%%%%%%%%%%%%%%%%%%%%%%%%%%%%%%%%%%

\def\ABAKUS{\begingroup\offinterlineskip

% Parameter (\"anderbar)
\dimen0=24pt            % Gesamtgr\"o{\ss}e
\dimen1=0.01\dimen0 % Strichdicke
%\dimen1=.8pt
\font\kugelfont=cmbsy5 scaled \magstephalf   % 1,2 facher cmbsy font
\def\kugel{{\kugelfont\char'16}}%

% Hilfsgr\"o{\ss}en
\dimen2=\dimen0\advance\dimen2by-6\dimen1\divide\dimen2by5%Spaltenbreite
\def\h{\hskip\dimen2}%
\dimen3=\dimen0\advance\dimen3by-3\dimen1\divide\dimen3by7%Zeilenh\"ohe
\setbox0\hbox to\dimen1{\hss\kugel\hss}%
\dimen4=\dimen3\advance\dimen4by-\ht0\advance\dimen4by-\dp0
  \divide\dimen4by2%Reststrichl\"ange \"uber und unter der Kugel
\def\restline{\hrule height\dimen4 depth0pt width\dimen1}%
\def\O{\vbox{\restline\copy0\restline}}%
\def\N{\vrule height\dimen3 depth0pt width\dimen1}%
% Abakus Tabelle
\vtop{\hrule height0pt depth\dimen1
      \hbox{\N\h\O\h\N\h\N\h\N\h\N}
      \hbox{\N\h\N\h\O\h\O\h\O\h\N}
      \hrule height\dimen1
      \hbox{\N\h\O\h\O\h\O\h\O\h\N}
      \hbox{\N\h\N\h\O\h\N\h\O\h\N}
      \hbox{\N\h\O\h\O\h\O\h\N\h\N}
      \hbox{\N\h\O\h\O\h\O\h\O\h\N}
      \hbox{\N\h\O\h\N\h\O\h\O\h\N}
      \hrule height\dimen1
     }
\endgroup}

%%%%%%%%%%%%%%%%%%%%%%%%%%%%%%%%%%%%%%%%%%%%%%%%%%%%%%%%%%%%%%%%%%%%%%%%%%%%%
%%%
%%% TUM Logo, use with \TUM{1cm}
%%%
%%%%%%%%%%%%%%%%%%%%%%%%%%%%%%%%%%%%%%%%%%%%%%%%%%%%%%%%%%%%%%%%%%%%%%%%%%%%%

\def\TUM#1{% von Gunnar Teege (leicht ge\"andert (siehe Kommentar))
%Argument gibt die Breite an
\dimen0=#1\dimen0=.018\dimen0%(war.012)% (war .018)
\dimen1=#1\dimen1=.1143\dimen1%
\dimen2=#1\dimen2=.419\dimen2%
\dimen3=#1\dimen3=.0857\dimen3%
%
\dimen4=\dimen1\advance\dimen4 by-\dimen0%
\setbox1=\vbox{\hrule width\dimen0 height\dimen4 depth0pt}%
\advance\dimen4 by\dimen2%
\setbox8=\vbox{\hrule width\dimen0 height\dimen4 depth0pt}%
\advance\dimen4 by-\dimen2\advance\dimen4 by-\dimen0%
\setbox4=\vbox{\hrule width\dimen4 height\dimen0 depth0pt}%
\advance\dimen4 by\dimen1\advance\dimen4 by\dimen3%
\setbox6=\vbox{\hrule width\dimen4 height\dimen0 depth0pt}%
\advance\dimen4 by\dimen3\advance\dimen4 by\dimen0%
\setbox9=\vbox{\hrule width\dimen4 height\dimen0 depth0pt}%
\advance\dimen4 by\dimen1%
\setbox7=\vbox{\hrule width\dimen4 height\dimen0 depth0pt}%
\dimen4=\dimen3%
\setbox5=\vbox{\hrule width\dimen4 height\dimen0 depth0pt}%
\advance\dimen4 by-\dimen0%
\setbox2=\vbox{\hrule width\dimen4 height\dimen0 depth0pt}%
\dimen4=\dimen2\advance\dimen4 by\dimen0%
\setbox3=\vbox{\hrule width\dimen0 height\dimen4 depth0pt}%
%
\setbox0=\vtop{\vskip0pt% (war \vbox und ohne \vskip0pt)
\hbox{\box9\lower\dimen2\copy3\lower\dimen2\copy5\lower\dimen2\copy3\box7}%
\kern-\dimen2\nointerlineskip%
\hbox{\raise\dimen2\box1\raise\dimen2\box2\copy3\copy4\copy3%
\raise\dimen2\copy5\copy3\box6\copy3\raise\dimen2\copy5\copy3\copy4\copy3%
\raise\dimen2\box5\box3\box4\box8}}%
\leavevmode\box0}


%%%%%%%%%%%%%%%%%%%%%%%%%%%%%%%%%%%%%%%%%%%%%%%%%%%%%%%%%%%%%%%%%%%%%%%%%%%%%
%%%
%%% TUM Header, use with \tumheader
%%%
%%%%%%%%%%%%%%%%%%%%%%%%%%%%%%%%%%%%%%%%%%%%%%%%%%%%%%%%%%%%%%%%%%%%%%%%%%%%%

\def\tumheader{
%\dimen5=9.1cm% \ / Diese drei Gr\"o{\ss}en
\dimen5=0.45\textwidth

\hbox to\textwidth{\ABAKUS\hfil\hbox to\dimen5{\vtop{\vskip0pt
    \hbox to\dimen5
        {\scriptsize TECHNISCHE\ UNIVERSIT{\kern-.1em}\"A{\kern-.1em}T\ M\"UNCHEN}
    \vskip2mm
    \hbox to\dimen5
        {\small FAKULT{\kern-.1em}\"A{\kern-.1em}T\ F\"UR\ INFORMATIK}
}}\hfil\TUM{1.5cm}}}


% title
\title{Garbage Collection \& Reference Counting}
\author{Memory-Management in Rust}
\institute[TUM]{
	\vspace{-15em}\tumheader\vspace{18em}
}
\date{Clemens Ruck \& Alex Egger \\ Summer Term 2017}

\begin{document}
\maketitle
\begin{frame}
	\frametitle{Inhalt}
	\begin{textblock}{6}(0,0)
		\begin{alertblock}{Arten des Memory-Managements}
			\begin{enumerate}
				\item Probleme bei manuellem Memory-Management
				\item Garbage Collection
				\item Rust's Ansatz
			\end{enumerate}
		\end{alertblock}
		\begin{alertblock}{Memory Management in Rust}
			\begin{enumerate}
				\item Stack-Allokation
				\item Heap-Allokation
				\item Reference Counting in Rust
				\item 'unsafe' und Raw Pointer
			\end{enumerate}
		\end{alertblock}
	\end{textblock}
\end{frame}
\begin{frame}[fragile]
	\frametitle{Manuelles Memory-Management}
	Dynamisch Speicher reservieren in C:
	\begin{minted}[tabsize=4, frame=single]{c}
void *malloc(size_t size);
	\end{minted}
	Speicher freigeben:
	\begin{minted}[tabsize=4, frame=single]{c}
void free(void *ptr);
	\end{minted}
	\vspace{1cm}
	\attention Es kann zu unnötigen Bugs führen, wenn vergessen wird, wann und wo schon Speicher freigegeben wurde.
\end{frame}
\begin{frame}[fragile]
	\frametitle{Probleme - Memory Leaks}
	\begin{minted}[tabsize=4, frame=single]{c}
int main(void) {
	while(1) {

		// Allocate some amount of memory on the heap.
		char *c;
		if(!(c = malloc(20 * sizeof(char)))) {
			perror("Could not allocate memory on heap.");
			return 1;
		}

		// Do something with allocated memory.

		// Memory is never freed, and can never be
		// reclaimed by the system.
		}
	return 0;
}
	\end{minted}
\end{frame}
\begin{frame}[fragile]
	\frametitle{Problems - Double Free}
	\begin{minted}[frame=single, tabsize=4]{c}
int main(void) {
	// Allocate some memory just like before.
	char *c = malloc(10);

	do_something(c);

	// Do something again, but the memory was
	// already freed!
	do_something(c);
}

void do_something(char *c) {
	// Do something with the memory here.

	// Then free the memory.
	free(c);
}
	\end{minted}
\end{frame}
\begin{frame}[fragile]
	\frametitle{Probleme - Use after Free}
	\begin{minted}[frame=single, tabsize=4]{c}
int main(void) {
	// Allocate memory.
	char *c = malloc(10);

	// Do something and then free the memory.
	free(c);

	// The memory now doesn't belong to us anymore, so
	// this will result in a segmentation fault.
	(*c)++;

	return 0;
}
	\end{minted}
\end{frame}
\begin{frame}
	\frametitle{Garbage Collection}
	\textbf{Garbage Collection} ist eine Form des automatischen Memory-Managements. Hierbei wird versucht Speicher freizugeben, der von Objekten benutzt wird, deren Lebenszeit abgelaufen ist.
\end{frame}
\begin{frame}
	\frametitle{Beispiel - Mark \& Sweep}
	\begin{figure}
		\centering
		\def\svgwidth{230pt}
		\input{svg/mark_sweep_setup.pdf_tex}
		\caption{Eine Graphen-Darstellung von lebenden Objekten.}
	\end{figure}
\end{frame}
\begin{frame}
	\frametitle{Beispiel - Mark \& Sweep}
	\begin{figure}
		\centering
		\def\svgwidth{230pt}
		\input{svg/mark_sweep_mark.pdf_tex}
		\caption{Die 'Mark'-Phase des Algorithmus.}
	\end{figure}
\end{frame}
\begin{frame}
	\frametitle{Beispiel - Mark \& Sweep}
	\begin{figure}
		\centering
		\def\svgwidth{230pt}
		\input{svg/mark_sweep_sweep.pdf_tex}
		\caption{Die 'Sweep'-Phase des Algorithmus.}
	\end{figure}
\end{frame}
\begin{frame}[fragile]
	\frametitle{Memory-Management in Rust}
	In Rust werden Variablen standardmäßig auf dem Stack allokiert:
	\begin{minted}[tabsize=4, frame=single]{rust}
let x = 12; // Auf dem Stack.
	\end{minted}
	\vspace{1em}
	Um Speicher auf dem Heap zu reservieren, benutzt man \mintinline{rust}{Box<T>}:
	\begin{minted}[tabsize=4, frame=single]{rust}
let y = Box::new(56); // Auf dem Heap.
	\end{minted}
\end{frame}
\begin{frame}[fragile]
	\frametitle{Variablen auf dem Stack}
	\begin{columns}[T, c]
		\begin{column}{0.48\textwidth}
			\begin{tabular}{| l | l | l |}
				\hline
				Adresse & Name & Wert\\ \hline
				0 & x & 42 \\ \hline
			\end{tabular}
		\end{column}
		\begin{column}{0.48\textwidth}
			\begin{minted}[escapeinside=||, linenos, tabsize=4, frame=single]{rust}
fn main() {
	|\colorbox{yellow}{let x = 42;}|
    other();

	//...
}

fn other() {
	let y = 27;
	let z = 99;
}
			\end{minted}
		\end{column}
	\end{columns}
\end{frame}
\begin{frame}[fragile]
	\frametitle{Variablen auf dem Stack}
	\begin{columns}[T, c]
		\begin{column}{0.48\textwidth}
			\begin{tabular}{| l | l | l |}
				\hline
				Adresse & Name & Wert \\ \hline
				2 & z & 99 \\ \hline
				1 & y & 27 \\ \hline
				0 & x & 42 \\ \hline
			\end{tabular}
		\end{column}
		\begin{column}{0.48\textwidth}
			\begin{minted}[escapeinside=||, linenos,tabsize=4,frame=single]{rust}
fn main() {
	let x = 42;
	other();

	//...
}

fn other() {
	let y = 27;
	|\colorbox{yellow}{let z = 99;}|
}
			\end{minted}
		\end{column}
	\end{columns}
\end{frame}
\begin{frame}[fragile]
	\frametitle{Variablen auf dem Stack}
	\begin{columns}[T, c]
		\begin{column}{0.48\textwidth}
			\begin{tabular}{| l | l | l |}
				\hline
				Adresse & Name & Wert \\ \hline
				0 & x & 42 \\ \hline
			\end{tabular}
		\end{column}
		\begin{column}{0.48\textwidth}
			\begin{minted}[escapeinside=||, linenos,tabsize=4,frame=single]{rust}
fn main() {
	let x = 42;
	other();

	|\colorbox{yellow}{\/\/...}|
}


fn other() {
	let y = 27;
	let z = 99;
}
			\end{minted}
		\end{column}
	\end{columns}
\end{frame}
\begin{frame}[fragile]
	\frametitle{Speicherreservierung auf dem Heap}
	\begin{columns}[T, c]
		\begin{column}{0.48\textwidth}
			\begin{tabular}{| l | l | l |}
				\hline
				Adresse & Name & Wert \\ \hline
				1 & y & ??? \\ \hline
				0 & x & 42 \\ \hline
			\end{tabular}
		\end{column}
		\begin{column}{0.48\textwidth}
			\begin{minted}[escapeinside=||, linenos, tabsize=4, frame=single]{rust}
fn main() {
	let x = 42;
	|\colorbox{yellow}{let y = Box::new(39);}|
}
			\end{minted}
		\end{column}
	\end{columns}
\end{frame}
\begin{frame}[fragile]
	\frametitle{Speicherreservierung auf dem Heap}
	\begin{columns}[T, c]
		\begin{column}{0.48\textwidth}
			\begin{tabular}{| l | l | l |}
				\hline
				Adresse & Name & Wert \\ \hline
				0xffff & & 39 \\ \hline
				... &   & \\ \hline
				1 & y & $->$ ffff \\ \hline
				0 & x & 42 \\ \hline
			\end{tabular}
		\end{column}
		\begin{column}{0.48\textwidth}
			\begin{minted}[escapeinside=||, linenos, tabsize=4, frame=single]{rust}
fn main() {
	let x = 42;
	|\colorbox{yellow}{let y = Box::new(39);}|
}
			\end{minted}
		\end{column}
	\end{columns}
\end{frame}
\begin{frame}
	\frametitle{Vergleich: Heap \& Stack}
	\begin{enumerate}
		\item Den Stack zu verwalten ist trivial
		\item Den Heap zu verwalten ist aufwendig (z.B. wegen Fragmentierung)
		\item Standardmäßig auf dem Stack zu arbeiten, lässt einfacher über die Lebenszeit von Variablen nachdenken
	\end{enumerate}
\end{frame}
\begin{frame}
	\frametitle{Reference Counting}
	Referenzen können als gerichteter Graph dargestellt werden, wobei die Knoten Objekte darstellen und zwischen zwei Knoten eine gerichtete Kante besteht, falls eines eine Referenz auf das andere hält.
	\begin{figure}
		\def\svgwidth{120pt}
		\LARGE
		\input{svg/rc_setup.pdf_tex}
	\end{figure}
\end{frame}
\begin{frame}
	\frametitle{Reference Counting}
	Wenn eine neue Referenz auf ein Objekt erstellt wird, wird der \textbf{Reference Counter} erhöht.
	\begin{figure}
		\def\svgwidth{120pt}
		\LARGE
		\input{svg/rc_new.pdf_tex}
	\end{figure}
\end{frame}
\begin{frame}
	\frametitle{Reference Counting}
	Wenn ein Objekt freigegeben wird, werden alle Referenzen die es hält freigegeben und deren jeweilige Referenzzähler dekrementiert.
	\begin{figure}
		\def\svgwidth{120pt}
		\LARGE
		\input{svg/rc_delete.pdf_tex}
	\end{figure}
\end{frame}
\begin{frame}
	\frametitle{Reference Counting}
	Objekte können nur freigegeben werden, wenn ihr \textbf{Referenzzähler auf 0} steht! Sonst werden die Referenzen, die noch auf das Objekt zeigen, ungültig.
	\begin{figure}
		\def\svgwidth{120pt}
		\LARGE
		\input{svg/rc_delete_bad.pdf_tex}
	\end{figure}
\end{frame}
\begin{frame}
	\frametitle{Limiterungen}
	\attention Zyklische Referenzen können nicht freigegeben werden!
	\begin{figure}
		\def\svgwidth{100pt}
		\input{svg/rc_cycle.pdf_tex}
	\end{figure}
	Dieses Problem kann durch den Einsatz eines Incremental Garbage Collectors gelöst werden, der spezifisch auf zyklische Referenzen ausgelegt ist.
	Eine andere Lösung ist es, zyklische Referenzen in den eigenen Datenstrukturen zu verbieten.
\end{frame}
\begin{frame}[fragile]
	\frametitle{\mintinline{rust}{Rc<T>} in Rust}
	In Rust können wir Reference Counting benutzen, indem wir das \mintinline{rust}{Rc<T>} Struct benutzen.
	\\
	Um es zu nutzen, muss zuerst \mintinline{rust}{Rc::new} aufgerufen werden:
	\begin{minted}[tabsize=4, frame=single]{rust}
let t = ...;
let rc_t = Rc::new(t);
	\end{minted}
	Um eine neue Referenz zu erstellen und den Referenzzähler zu erhöhen, wird \mintinline{rust}{Rc.clone()} benutzt:
	\begin{minted}[tabsize=4,frame=single]{rust}
let second_rc_t = rc_t.clone();
	\end{minted}
	\attention Um Reference Counting in Szenarien mit multiplen Threads zu benutzen, muss man \mintinline{rust}{Arc<T>} benutzen, was inperformanter ist.
\end{frame}
\begin{frame}[fragile]
	\frametitle{Reference Counting - Beispiel}
	\begin{minted}[tabsize=4, frame=single]{rust}
struct Owner {
	name: String,
	// Fields...
}

struct Car {
	owner: Rc<Owner>,
	// Fields...
}
	\end{minted}
\end{frame}
\begin{frame}[fragile]
	\frametitle{Reference Counting - Beispiel}
	\begin{minted}[tabsize=4, frame=single]{rust}
fn main() {
	let owner = Rc::new(Owner{
		name: "Lars",
		// Fields...
	});

	let car = Car {
		owner: owner.clone(),
		// Fields...
	};

	let car2 = Car {
		owner: owner.clone(),
		// Fields...
	};
	// ...
}
	\end{minted}
\end{frame}
\begin{frame}[fragile]
	\frametitle{Reference Counting - Beispiel}
	\begin{minted}[tabsize=4, frame=single]{rust}
fn main() {
	// ...

	// Drop die lokale Variable 'owner'
	drop(owner);

	// Funktionert immer noch, da es immernoch
	// Referenzen auf owner gibt!
	println!("{}", car.owner.name);
}
	\end{minted}
\end{frame}

\begin{frame}
	\frametitle{'unsafe'}
	Das \mintinline{rust}{unsafe} Keyword erlaubt:
	\begin{enumerate}
		\item Das Benutzen und Updaten von static mutable Variablen.
		\item Das Dereferenzieren von Raw Pointern.
		\item Das Aufrufen von als 'unsafe' markierten Funktionen.
	\end{enumerate}
\end{frame}
\begin{frame}[fragile]
	\frametitle{Raw Pointer}
	Es gibt zwei Typen von Raw Pointern:
	\begin{enumerate}
		\item \mintinline{rust}{*const T}
		\item \mintinline{rust}{*mut T}
	\end{enumerate}
	\vspace{1cm}
	Raw Pointer sind einfache C-artige Zeiger auf einen Speicherbereich. Dabei unterliegen sie nicht den üblichen Ownership- oder Lifetimeregeln. Sie können nur in "unsafen" Codebereichen dereferenziert werden.
\end{frame}
\begin{frame}[fragile]
	\frametitle{Example}
	\begin{minted}[linenos, tabsize=4, frame=single]{rust}
fn main() {
	let i: u32 = 77;

	// Einen Raw Pointer erstellen,
	// kann man auch in safem Code.
	let x = &i as *const u32;

	// Derefenzieren führt allerdings zu einem Fehler.
	let y = *x;

	// Wenn der Codeblock als unsafe markiert ist,
	// ist auch derefenzieren erlaubt.
	unsafe {
		let z = *x;
	}
}
	\end{minted}
\end{frame}
\begin{frame}
	\frametitle{Quellen}
	\begin{enumerate}
		\item https://doc.rust-lang.org/book/first-edition/unsafe.html
		\item https://doc.rust-lang.org/book/first-edition/choosing-your-guarantees.html
		\item ftp://ftp.cs.utexas.edu/pub/garbage/gcsurvey.ps
		\item https://doc.rust-lang.org/std/
	\end{enumerate}
\end{frame}
\end{document}
